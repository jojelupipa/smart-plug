\documentclass[a4paper,10pt]{article}

% Lenguaje
\usepackage[spanish]{babel}
\usepackage[utf8]{inputenc}
\setlength{\parindent}{0pt}

% Ajustes documento
\usepackage{geometry}
\geometry{left=3cm,right=3cm,top=3cm,bottom=3cm,headheight=1cm,headsep=0.5cm}
\usepackage{enumitem}

\usepackage{xcolor}
\usepackage{hyperref}
\hypersetup{
    colorlinks=true,
    linkcolor=[rgb]{0 0 0},
    filecolor=magenta,      
    urlcolor=blue,
}
\urlstyle{same}

\usepackage{amsmath}

% Sangrado y espacio entre párrafos
\setlength{\parskip}{1em}
\setlength{\parindent}{1.5em}

\begin{document}

\tableofcontents

\hypersetup{
    colorlinks=true,
    linkcolor=[rgb]{0.149 0 0.3216},
    filecolor=magenta,      
    urlcolor=blue,
    citecolor=blue,
}
\urlstyle{same}


\newpage

\section{Internet de las Cosas}\label{internet-de-las-cosas}

Uno de los términos más recurrentes en el ámbito de la tecnología a día
de hoy es el de \emph{``Internet de las Cosas''} (en inglés, IoT, \emph{Internet of
Things}). Pero, puesto que el término puede resultar ambiguo, es
conveniente dar una definición sobre esta que permita esclarecer el
concepto.

El Internet de las Cosas es un paradigma tecnológico en sí mismo.
Desglosándo el concepto vemos que reúne dos tecnologías. Por un lado
parte de ``las cosas'', que abarca desde los dispositivos y
electrodomésticos que podemos encontrar en nuestro día a día; que tienen
una funcionalidad en sí misma (p.ej. una televisión, un frigorífico),
hasta incluso podemos abstraer a personas o animales (una persona con un
marcapasos, un ave con un geolocalizador u otros ejemplos\cite{iotagendawebsiteWhatInternetThings}). A estas ``cosas'' se les añade el término
``Internet'', que no es más que una abstracción de la conectividad que
permite dotar a las cosas de comunicación con el exterior, extendiendo
sus funcionalidades sin privarlas de su cometido original. Esta es
su principal característica, la posibilidad de comunicación con otras
``cosas'' mediante internet y sin la necesidad de intervención humana.

La idea que trasciende de esto es que cualquier cosa o dispositivo que
usemos habitualmente puede conectarse a una red, a internet. Haciendo
que, en términos de redes, un dispositivo IoT se pueda equiparar a un
ordenador convencional.

\subsection{¿Qué es un dispositivo
IoT?}\label{quuxe9-es-un-dispositivo-iot}

En el símil anterior hemos adelantado el concepto de dispositivo IoT sin
llegar a definirlo completamente. A continuación explicaremos qué es
concretamente un dispositivo IoT y qué lo diferencia de otras ``cosas''.

Un dispositivo IoT es una entidad (objeto, o incluso animales o
personas) que tiene una funcionalidad en sí misma y a la cual dotamos de
una capacidad de conexión y telecomunicación. De modo que sea capaz de,
por sí misma, comunicarse con otros dispositivos en su entorno con esta
misma capacidad.

Los beneficios que podemos obtener de este paradigma se pueden aplicar
en muchas áreas, por
ejemplo\cite{vongsingthongs.;smanchats.IoTExamplesSuranaree}:

\begin{itemize}
\item
  Logística: Rastreo de envíos por correo o estado del stock en
  almacenes automatizando lectura de los items mediante wifi.
\item
  Transporte: Gestión automática de rutas por GPS; captura y procesado
  de infracciones de velocidad.
\item
  Salud: Desde una pulsera que capte tus hábitos de vida hasta
  la supervisión de personas ancianas que viven solas.
\item
  Monitorización: Desde instalación de sensores en un bosque para medir
  datos de contaminación hasta medir en un hogar el consumo de un
  electrodoméstico.
\end{itemize}

Este último ejemplo es justo la motivación de nuestro trabajo. Podemos
usar esta tecnología para crear un enchufe inteligente que nos permita
saber nuestro consumo eléctrico y controlar el uso que le damos.

\newpage

\section{Revisión de enchufes inteligentes existentes en el
mercado}\label{revisiuxf3n-de-enchufes-inteligentes-existentes-en-el-mercado}

\subsection{¿Qué es un enchufe? Tipos de
enchufe}\label{quuxe9-es-un-enchufe-tipos-de-enchufe}

Un enchufe es una fuente de alimentación entre sistemas eléctricos. En
este proyecto trataremos los enchufes dentro del ámbito doméstico o
comercial frente al ámbito industrial, que tiende a trabajar con
voltajes de un orden muy superior.

Un enchufe en el ámbito doméstico permite la conexión entre un
dispositivo electrónico y la corriente alterna para la alimentación de
este dispositivo.

Estos trabajan con unos voltajes entre 100V y 240V
\cite{iecIECWorldPlugs}. El tipo de enchufe, voltaje y frecuencia
usados en una región geográfica vienen determinados por un convenio
fijado por el gobierno de esa zona.

Por razones históricas\cite{nuevatribunaOrigenFrecuenciasElectricas}
no existe una estandarización.

Trabajaremos con el tipo de enchufe usado en España, que cuenta con un
potencial de 230V y una frecuencia de 50Hz\cite{IECWorldPlugs}.


\subsection{¿Qué es un enchufe
inteligente?}\label{quuxe9-es-un-enchufe-inteligente}

Un enchufe inteligente extiende esta definición de enchufe. Actuando
como una interfaz que permite añadir un amplio abanico de
funcionalidades relativas al manejo y supervisión de estos dispositivos
externos. Dotándole de las ventajas de la conectividad.

Esto encaja con la definición de dispositivo IoT que mencionábamos
previamente. Un objeto cotidiano que ha sido dotado de la capacidad de
comunicación con otros dispositivos.

\subsection{Funcionalidades comunes que ofertan los enchufes
electrónicos
actualmente}\label{funcionalidades-comunes-que-ofertan-los-enchufes-electruxf3nicos-actualmente}

\subsubsection{Monitorización:}\label{monitorizaciuxf3n}

Supervisar el uso de energía: Esto permite establecer un control sobre
su uso, realizar estimaciones sobre el coste o incluso notificar frente
a anomalías como el consumo en horarios no esperados o picos de voltaje.

\subsubsection{Control:}\label{control}

Manejar tus dispositivos de manera remota, programar el horario de
funcionamiento de estos o la manera en la que funcionan son solo algunas
de las opciones. Destacando la posibilidad de integración con otros
dispositivos que proporcionen una mayor accesibilidad (p.ej. la interfaz
de voz de un asistente de móvil).

\subsection{Tipos de enchufes}\label{tipos-de-enchufes}

Atendiendo a sus características más distintivas podemos hacer una
clasificación (no excluyente) de:

\begin{itemize}
\item
  Enchufes inteligentes (E.I.) por control remoto (wifi, bluetooth,
  infrarrojos u otros tipos de señales electromagnéticas)
\item
  E.I. programables
\item
  E.I. de regletas
\end{itemize}

\subsection{Modelos en el mercado}\label{modelos-en-el-mercado}

La comercialización de este producto está muy extendida, proporcionando
una amplia oferta de productos los cuales podemos recogerlos bajo la
clasificación previa:

\begin{itemize}
\item
  Programación temporal: Permiten la ejecución de tareas a horas
  concretas del día. Por ejemplo:
  \href{https://web.archive.org/web/20191111121153/https://www.amazon.es/Garza-Power-Temporizador-anal\%C3\%B3gico-programaci\%C3\%B3n/dp/B00URUVDW2/}{Garza
  400603, analógico},
  \href{https://web.archive.org/web/20191112121450/https://www.amazon.es/dp/B00ZJ1LQDK}{Orbegozo
  PG 20, digital}
\item
  Control remoto: Se pueden controlar con mando a distancia, a través de
  una app o con control por voz. Por ejemplo:
  \href{https://web.archive.org/web/20191112130933/https://www.amazon.es/dp/B06W586CDZ}{TP-Link
  HS100} o la
  \href{https://web.archive.org/web/20191116174434/https://www.amazon.es/dp/B07DJ2G1CW}{Regleta
  Xiaomi}, que es un ejemplo de intersección entre E.I. de regleta y por
  control remoto.
\end{itemize}

\newpage

\section{Arquitecturas IoT}\label{arquitecturas-iot}

El paradigma de IoT viene a integrarse entre los mecanismos
convencionales y el resto de redes. En sus primeros pasos se adaptaron
sus diseños a las necesidades de estos sistemas. Pero según fue
desarrollándose fueron naciendo algunas arquitecturas que ayudaban a
definir y a trabajar mejor con los requisitos de estos nuevos sistemas.

Existen diferentes arquitecturas en la actualidad, que difieren en el
enfoque que le dan a ciertos campos. Dos de los más conocidos son IoT-A
e IIRA. Siendo el primero el más extendido desde su lanzamiento en 2012
\cite{weyrichReferenceArchitecturesInternet2016}.

Algunos campos de comparación entre arquitecturas de IoT son\cite{atzoriInternetThingsSurvey2010}:

\begin{itemize}
\item
  El enfoque particular a casos de negocio frente a conceptos generales.
\item
  Orientación a Internet: Si se introducen nuevos protocolos y se
  adaptan los ya existentes a sus necesidades.
\item
  Tipo de dispositivo sobre el cual se orienta (como sensores o
  actuadores).
\end{itemize}

Estas arquitecturas coinciden en la definición de capas de manera
análoga a TCP/IP u
OSI\cite{weyrichReferenceArchitecturesInternet2016}, pero adaptándolas
a las necesidades del IoT. Esta conjunto de capas se pueden
generalizar en 3 capas principales: Percepción, asociada a los
dispositivos físicos, consistente en sensores; capa de red, encargada
de transmitir información entre sensores y sistemas procesadores
finales; y capa de aplicación, para el proceso final de la
información\cite{khanFutureInternetInternet2012}.

Para conocer cómo se produce la comunicación a distintos niveles de la
arquitectura es necesario analizar sus protocolos.

\subsection{Protocolos}\label{subsec:protocolos}

Dos ordenadores convencionales necesitan una pila de protocolos para que
puedan transmitir información a las distintas capas que conforman la
comunicación. Ante los requisitos que surgían con los dispositivos IoT,
emergían también protocolos que daban respuesta a estos. Estableciéndose
así una pila de protocolos de red especializada para IoT.

A fin de permitir estas comunicaciones de manera liviana es necesario
alguna tecnología a \textbf{nivel físico} que pueda solventar el
problema de la saturación del espectro de frecuencias, algunas
propuestas se basan en aprovechar el ruido blanco (frecuencias no
usadas en el espectro de radio)\cite{tempertonTVWhiteSpace2015}.  En
la \textbf{capa de enlace} se encuentra 6LoWPAN
\cite{schumacherIPv6LowPowerWireless} para conectar los dispositivos a
una red IP, dicha red ha de ser del tipo IPv6 (\textbf{capa de red})
para dar soporte a la escalabilidad.

En la \textbf{capa de aplicación} existen alternativas para
proporcionar un transporte de datos con baja sobrecarga en la
comunicación. Algunas de estas opciones\cite{al-fuqahaInternetThingsSurvey2015}
son: \textbf{CoAP}
(Constrained Application Protocol), usado como un protocolo web basado
en \textit{REST}\cite{WebServicesArchitecture}; \textbf{XMPP}
(Extensible Messaging and Presence Protocol), usado especialmente en
mensajería; \textbf{DDS} (Data Distribution Service), especializado en
aplicaciones de tiempo real y con alta disponibilidad; o \textbf{MQTT}
(Message Queue Telemetry Transport), para conectar dispositivos
embebidos con aplicaciones y middleware.

Esta es una de las decisiones que más afecta al diseño, pues el resto
del diseño de la arquitectura se ve afectado por ello.

\subsection{Modelos de interconexión de
red}\label{modelos-de-interconexiuxf3n-de-red}

De acuerdo a la topología de la red, el modo en el que se conectan los
dispositivos involucrados, reconocen ocho tipos básicos de modelos de
interconexión\cite{bicsiNetworkDesignBasics2002}.

\begin{itemize}
\item{Punto a punto: Dos dispositivos finales se perciben directamente
  conectados entre sí.}
\item{Cadena Margarita: Se conecta en serie cada computador al
  siguiente, los mensajes se retransmiten durante toda la cadena.}
\item{Bus: Todos los dispositivos están conectados entre sí usando
  un cable central. Esto implica una difusión de cada mensaje a
  todos los nodos. Es más usual en redes locales.}
\item{Estrella: Un conjunto de nodos periféricos se conectan con un
  nodo central que actúa como servidor.}
\item{Anillo: Similar a una topología de “cadena margarita” donde
  existe un único bucle y la comunicación se realiza en un único
  sentido.}
\item{Malla: Los dispositivos se conectan de muchos a muchos, puede
  ser completamente conectada o parcialmente conectada si todos los
  nodos se encuentran conectados entre sí o no.}
\end{itemize}

\newpage

\section{Diseño de la arquitectura de nuestro
sistema}\label{diseuxf1o-de-la-arquitectura-de-nuestro-sistema}

\subsection{Descripción general}\label{descripciuxf3n-general}

Nuestro enchufe inteligente recogerá mediante el sensor SCT013 la
corriente que requiere el dispositivo conectado. En el módulo ESP32 se
realiza el cálculo del consumo en Vatios y se emite hacia nuestro
servidor, una Raspberry Pi con Mosquitto, y MQTT, de donde se podrá
transmitir a una aplicación externa para supervisar el consumo y
manipular el enchufe.

\subsection{Modelo de red usado}\label{modelo-de-red-usado}

La topología de red usada en este proyecto es el modelo en
estrella. En el centro se encuentra nuestro servidor que se encarga de
gestionar tanto la información de los sensores como de otorgar la
visualización y manejo sobre los enchufes a los usuarios. En los
extremos se encontrarían tanto los enchufes inteligentes como los
usuarios desde sus aplicaciones.

\subsection{Protocolo MQTT}\label{protocolo-mqtt}

De los protocolos previamente mencionados [\ref{subsec:protocolos}]
optamos por utilizar MQTT por su facilidad de integración con
Mosquitto, un broker (agente intermediario) que implementa este
protocolo y cuya ligera carga le permite ejecutarse en servidores de
bajas prestaciones (Raspberry Pi, en nuestro caso).

MQTT utiliza el patrón de publicación/suscripción (publish-suscribe),
un publicador escribe sobre un tema (topic) del cual un suscriptor
lee. Está construido sobre TCP.

El tipo de comunicación en este protocolo es indicado en cada mensaje
en la cabecera del paquete. Proporciona mensajes para conexión
(CONNECT), desconexión (DISCONNECT), publicar en un tema (PUBLISH),
suscribirse/desuscribirse a un tema (SUBSCRIBE/UNSUSCRIBE) y otros
mensajes de control correspondientes a una respuesta ACK (pues está
basado en TCP)\cite{banksMQTTVersionEdited}.

\subsection{Diseño broker, publishers,
suscribers}\label{diseuxf1o-broker-publishers-suscribers}

Aunque existen otros subtipos de patrones publicador/suscriptor
basados en tipo o en contenido\cite{p.th.eugsterManyFacesPublish},
MQTT utiliza un modelo basado en temas.

Las principales componentes de este patrón son suscriptor, publicador
y el broker. Un dispositivo interesado se registra como suscriptor de
un tema y el broker se encargará de informarle cuando se publique algo
en el tema. El publicador actúa como un generador de información de
algún tema de interés. El broker es el intermediario que transmite a
los interesados, teniendo en cuenta además comprobaciones de seguridad
mediante autorización de publicadores y
suscriptores.\cite{al-fuqahaInternetThingsSurvey2015,hunkelerMQTTSPublishSubscribe2008}

\subsubsection{Mosquitto}

El broker que usamos en nuestro diseño es
\textbf{Mosquitto}\cite{EclipseMosquitto}. Agente que implementa MQTT
v3.1.1, la versión de MQTT usada en este proyecto.

\newpage

\bibliographystyle{unsrt}
\bibliography{biblio} 

\end{document}
