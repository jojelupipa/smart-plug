\documentclass[a4paper,10pt]{article}

% Lenguaje
\usepackage[spanish]{babel}
\usepackage[utf8]{inputenc}
\setlength{\parindent}{0pt}

% Ajustes documento
\usepackage{geometry}
\geometry{left=3cm,right=3cm,top=3cm,bottom=3cm,headheight=1cm,headsep=0.5cm}
\usepackage{enumitem}

\usepackage{hyperref}

\usepackage{amsmath}
\begin{document}

\section{Internet de las Cosas}\label{internet-de-las-cosas}

Uno de los términos más recurrentes en el ámbito de la tecnología a día
de hoy es el de ``Internet de las Cosas'' (en inglés, IoT, Internet of
Things). Pero, puesto que el término puede resultar ambiguo, es
conveniente dar una definición sobre esta que permita esclarecer el
concepto.

El Internet de las Cosas es un paradigma tecnológico en sí mismo.
Desglosándo el concepto vemos que reúne dos tecnologías. Por un lado
parte de ``las cosas'', que abarca desde los dispositivos y
electrodomésticos que podemos encontrar en nuestro día a día; que tienen
una funcionalidad en sí misma (p.ej. una televisión, un frigorífico)
hasta incluso podemos abstraer a personas o animales (una persona con un
marcapasos o un ave con un geolocalizador,
\href{https://internetofthingsagenda.techtarget.com/definition/Internet-of-Things-IoT}{ejemplos que podemos encontrar aquí}
). A estas ``cosas'' se les añade el término
``Internet'', que no es más que una abstracción de la conectividad que
permite dotar a las cosas de comunicación con el exterior, extendiendo
sus funcionalidades sin privarlas de su funcionamiento original. Esta es
su principal característica, la posibilidad de comunicación con otras
``cosas'' mediante internet y sin la necesidad de intervención humana.

La idea que trasciende de esto es que cualquier cosa o dispositivo que
usemos habitualmente puede conectarse a una red, a internet. Haciendo
que, en términos de redes, un dispositivo IoT se pueda equiparar a un
ordenador convencional.

\subsection{¿Qué es un dispositivo
IoT?}\label{quuxe9-es-un-dispositivo-iot}

En el símil anterior hemos adelantado el concepto de dispositivo IoT sin
llegar a definirlo completamente. A continuación explicaremos qué es
concretamente un dispositivo IoT y qué lo diferencia de otras ``cosas''.

Un dispositivo IoT es una entidad (objeto, o incluso animales o
personas) que tiene una funcionalidad en sí misma y a la cual dotamos de
una capacidad de conexión y telecomunicación. De modo que sea capaz de,
por sí misma, comunicarse con otros dispositivos en su entorno con esta
misma capacidad.

Los beneficios que podemos obtener de este paradigma se pueden aplicar
en muchas áreas, por
ejemplo\href{http://ird.sut.ac.th/e-journal/Journal/suwimonv/1403739/1403739.pdf}{p
5-17}:

\begin{itemize}
\item
  Logística: Rastreo de envíos por correo o estado del stock en
  almacenes, automatizando lectura de los ítems mediante wifi.
\item
  Transporte: Gestión automática de rutas por GPS; captura y procesado
  de infracciones de velocidad.
\item
  Salud: Desde una pulsera que capte tus hábitos de vida hasta
  monitorización de personas ancianas que viven solas.
\item
  Monitorización: Desde instalación de sensores en un bosque para medir
  datos de contaminación hasta medir en un hogar el consumo de un
  electrodoméstico.
\end{itemize}

Este último ejemplo es justo la motivación de nuestro trabajo. Podemos
usar esta tecnología para crear un enchufe inteligente que nos permita
saber nuestro consumo eléctrico y controlar el uso que le damos.

\section{Revisión de enchufes inteligentes existentes en el
mercado}\label{revisiuxf3n-de-enchufes-inteligentes-existentes-en-el-mercado}

\subsection{¿Qué es un enchufe? Tipos de
enchufe}\label{quuxe9-es-un-enchufe-tipos-de-enchufe}

Un enchufe es una fuente de alimentación entre sistemas eléctricos. En
este proyecto trataremos los enchufes dentro del ámbito doméstico o
comercial frente al ámbito industrial, que tiende a trabajar con
voltajes de un orden muy superior.

Un enchufe en el ámbito doméstico permite la conexión entre un
dispositivo electrónico y la corriente alterna para la alimentación de
este dispositivo.

Estos trabajan con unos voltajes entre 100V y 240V
(https://www.iec.ch/worldplugs/list\_byelectricpotential.htm). El tipo
de enchufe, voltaje y frecuencia usados en una región geográfica vienen
determinados por un convenio fijado por el gobierno de esa zona.

Por razones históricas
(https://www.nuevatribuna.es/articulo/ciencia/origen-frecuencias-electricas-50-y-60-hz/20150811110350118983.html)
no existe una estandarización.

Trabajaremos con el tipo de enchufe usado en España, que cuenta con un
potencial de 230V y una frecuencia de 50Hz.
(https://www.iec.ch/worldplugs/list\_bylocation.htm).

\subsection{¿Qué es un enchufe
inteligente?}\label{quuxe9-es-un-enchufe-inteligente}

Un enchufe inteligente extiende esta definición de enchufe. Actuando
como una interfaz que permite añadir un amplio abanico de
funcionalidades relativas al manejo y supervisión de estos dispositivos
externos. Dotándole de las ventajas de la conectividad.

Esto encaja con la definición de dispositivo IoT que mencionábamos
previamente. Un objeto cotidiano que ha sido dotado de la capacidad de
comunicación con otros dispositivos.

\subsection{Funcionalidades comunes que ofertan los enchufes
electrónicos
actualmente}\label{funcionalidades-comunes-que-ofertan-los-enchufes-electruxf3nicos-actualmente}

\subsubsection{Monitorización:}\label{monitorizaciuxf3n}

Supervisar el uso de energía: Esto permite establecer un control sobre
su uso, realizar estimaciones sobre el coste o incluso notificar frente
a anomalías como el consumo en horarios no esperados o picos de voltaje.

\subsubsection{Control:}\label{control}

Manejar tus dispositivos de manera remota, programar el horario de
funcionamiento de estos o la manera en la que funcionan son solo algunas
de las posibilidades. Destacando la posibilidad de integración con otros
dispositivos que proporcionen una mayor accesibilidad (p.ej. la interfaz
de voz de un asistente de móvil).

\subsection{Tipos de enchufes}\label{tipos-de-enchufes}

Atendiendo a sus características más distintivas podemos hacer una
clasificación (no excluyente) de:

\begin{itemize}
\item
  Enchufes inteligentes (E.I.) por control remoto (wifi, bluetooth,
  infrarojos u otros tipos de señales electromagnéticas)
\item
  E.I. programables
\item
  E.I. de regletas
\end{itemize}

\subsection{Modelos en el mercado}\label{modelos-en-el-mercado}

La comercialización de este producto está muy extendida, proporcionando
una amplia oferta de productos los cuales podemos recogerlos bajo la
clasificación previa:

\begin{itemize}
\item
  Programación temporal: Permiten la ejecución de tareas a horas
  concretas del día. Por ejemplo:
  \href{https://web.archive.org/web/20191111121153/https://www.amazon.es/Garza-Power-Temporizador-anal\%C3\%B3gico-programaci\%C3\%B3n/dp/B00URUVDW2/}{Garza
  400603, analógico},
  \href{https://web.archive.org/web/20191112121450/https://www.amazon.es/dp/B00ZJ1LQDK}{Orbegozo
  PG 20, digital}
\item
  Control remoto: Se pueden controlar con mando a distancia, a través de
  una app o con control por voz. Por ejemplo:
  \href{https://web.archive.org/web/20191112130933/https://www.amazon.es/dp/B06W586CDZ}{TP-Link
  HS100} o la
  \href{https://web.archive.org/web/20191116174434/https://www.amazon.es/dp/B07DJ2G1CW}{Regleta
  Xiaomi}, que es un ejemplo de intersección entre E.I. de regleta y por
  control remoto.
\end{itemize}

\section{Arquitecturas IoT}\label{arquitecturas-iot}

El paradigma de IoT viene a integrarse entre los mecanismos
convencionales y el resto de redes. En sus primeros pasos se adaptaron
sus diseños a las necesidades de estos sistemas. Pero según fue
desarrollándose fueron naciendo algunas arquitecturas que ayudaban a
definir y a trabajar mejor con los requisitos de estos nuevos sistemas.

Existen diferentes arquitecturas en la actualidad, que difieren en el
enfoque que le dan a ciertos campos. Dos de los más conocidos son IoT-A
e IIRA. Siendo el primero el más extendido desde su lanzamiento en 2012
\href{https://web.archive.org/web/20191116195145/https://www.researchgate.net/publication/288855901_Reference_Architectures_for_the_Internet_of_Things}{}.

Estos campos de comparación
son:\href{https://web.archive.org/web/20191116203648/https://www.sciencedirect.com/science/article/abs/pii/S1389128610001568?via=ihub}{}

\begin{itemize}
\item
  El enfoque particular a casos de negocio frente a conceptos generales.
\item
  Orientación a Internet: Se introducen nuevos protocolos y se adaptan
  los ya existentes a sus necesidades.
\item
  Tipo de dispositivo sobre el cual se orienta (como sensores o
  actuadores).
\end{itemize}

\subsection{Modelos de interconexión de
red}\label{modelos-de-interconexiuxf3n-de-red}

\subsection{Protocolos}\label{protocolos}

Dos ordenadores convencionales necesitan una pila de protocolos para que
puedan transmitir información a las distintas capas que conforman la
comunicación. Ante los requisitos que surgían con los dispositivos IoT,
surgían también protocolos que daban respuesta a estos. Estableciéndose
así una pila de protocolos de red especializada para IoT.

\section{Diseño de la arquitectura de nuestro
sistema}\label{diseuxf1o-de-la-arquitectura-de-nuestro-sistema}

\subsection{Descripción general}\label{descripciuxf3n-general}

\subsection{Modelo de red usado}\label{modelo-de-red-usado}

\subsection{Protocolo MQTT}\label{protocolo-mqtt}

\subsection{Diseño broker, publishers,
suscribers}\label{diseuxf1o-broker-publishers-suscribers}



\newpage

\cite{IECWorldPlugs}

%\nocite{*}


\bibliographystyle{abbrv}
\bibliography{biblio} 

\end{document}
